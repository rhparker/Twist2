% \documentclass[reprint, amsmath,amssymb,aps,pra]{revtex4-2}
\documentclass[11pt,reqno]{amsart}

\usepackage[pdfborder={0 0 0.5 [3 2]}, plainpages=false]{hyperref}%
\usepackage[left=1in,right=1in,top=1in,bottom=1in]{geometry}%
% \usepackage[shortalphabetic]{amsrefs}%
\usepackage{amsmath}
\usepackage{enumerate}
% \usepackage{enumitem}
\usepackage{amssymb}                
\usepackage{amsmath}                
\usepackage{amsfonts}
\usepackage{amsthm}
\usepackage{bbm}
\usepackage[table,xcdraw]{xcolor}
% \usepackage{float}
\usepackage{booktabs}
\usepackage{svg}
\usepackage{mathtools}
\usepackage{cool}
\usepackage{url}
\usepackage{graphicx,epsfig}
\usepackage{makecell}
\usepackage{array}

\usepackage[capitalize,nameinlink]{cleveref}
% Per SIAM Style Manual, "section" should be lowercase
\crefname{section}{section}{sections}
\crefname{subsection}{subsection}{subsections}
\Crefname{section}{Section}{Sections}
\Crefname{subsection}{Subsection}{Subsections}

% Per SIAM Style Manual, "Figure" should be spelled out in references
\Crefname{figure}{Figure}{Figures}

% Per SIAM Style Manual, don't say equation in front on an equation.
\crefformat{equation}{\textup{#2(#1)#3}}
\crefrangeformat{equation}{\textup{#3(#1)#4--#5(#2)#6}}
\crefmultiformat{equation}{\textup{#2(#1)#3}}{ and \textup{#2(#1)#3}}
{, \textup{#2(#1)#3}}{, and \textup{#2(#1)#3}}
\crefrangemultiformat{equation}{\textup{#3(#1)#4--#5(#2)#6}}%
{ and \textup{#3(#1)#4--#5(#2)#6}}{, \textup{#3(#1)#4--#5(#2)#6}}{, and \textup{#3(#1)#4--#5(#2)#6}}

% But spell it out at the beginning of a sentence.
\Crefformat{equation}{#2Equation~\textup{(#1)}#3}
\Crefrangeformat{equation}{Equations~\textup{#3(#1)#4--#5(#2)#6}}
\Crefmultiformat{equation}{Equations~\textup{#2(#1)#3}}{ and \textup{#2(#1)#3}}
{, \textup{#2(#1)#3}}{, and \textup{#2(#1)#3}}
\Crefrangemultiformat{equation}{Equations~\textup{#3(#1)#4--#5(#2)#6}}%
{ and \textup{#3(#1)#4--#5(#2)#6}}{, \textup{#3(#1)#4--#5(#2)#6}}{, and \textup{#3(#1)#4--#5(#2)#6}}

% Make number non-italic in any environment.
\crefdefaultlabelformat{#2\textup{#1}#3}

\def\noi{\noindent}
\def\T{{\mathbb T}}
\def\R{{\mathbb R}}
\def\N{{\mathbb N}}
\def\C{{\mathbb C}}
\def\Z{{\mathbb Z}}
\def\P{{\mathbb P}}
\def\E{{\mathbb E}}
\def\Q{\mathbb{Q}}
\def\ind{{\mathbb I}}
\def\calI{{\mathcal I}}
\def\calL{{\mathcal L}}
\def\Lw{{\mathcal{L}_\omega}}

\DeclareMathOperator{\spn}{span}
\DeclareMathOperator{\ran}{range}
\DeclareMathOperator{\diag}{diag}
\DeclareMathOperator{\md}{mod}

\newtheorem{lemma}{Lemma}
\newtheorem{theorem}{Theorem}
\newtheorem{corollary}{Corollary}
\newtheorem{definition}{Definition}
\newtheorem{proposition}{Proposition}
\newtheorem{hypothesis}{Hypothesis}

\title{Spatiotemporal dynamics of a twisted waveguide array}


\begin{document}

\maketitle

\section{Model}

Consider coupled mode equations for $m$ waveguides in ring with temporal dispersion term included, coupling parameter $k$, twist parameter $\phi$.
\begin{align*}
&i\partial_z c_n + \partial_t^2 c_n + k\left(e^{i\phi}c_{n-1}+e^{-i\phi}c_{n+1}\right)+|c_n|^2 c_n = 0 && n = 1, \dots, m,
\end{align*}
where the subscripts $n$ are the labels for the waveguides, and are taken $\md m$ to account for the circular arrangement.

Standing wave solutions of form $c_n e^{i \omega z}$ with frequency $\omega$ satisfy
\begin{align}\label{eq:standingwave}
\partial_t^2 c_n + k\left(e^{i\phi}c_{n-1}+e^{-i\phi}c_{n+1}\right)+|c_n|^2 c_n - \omega c_n = 0.
\end{align}
When the coupling parameter $k=0$ (i.e. we are at AC limit), solution at each node is either 0 or the standard NLS soliton
\begin{equation}\label{eq:NLSsoliton}
\psi(t) = \sqrt{2 \omega} \sech(\sqrt{\omega} t).
\end{equation}

We are interested in the case where the bulk of the intensity is contained a single node, which we label $n=0$. We will also take the number of waveguides $m$ to be even, although we will comment about what occurs when $m$ is odd. We label the opposite node in the ring by $n=m/2$, and the remaining node are labeled $\pm n$, for $n = 1, \dots, m/2-1$ [FIGURE]. We take the following ansatz for $c_n(t)$
\begin{equation}\label{eq:cnansatz}
c_n(t) = a_n(t)e^{i \theta_n(t)},
\end{equation}
where we have separated the amplitude $a_n(t)$ and the phase $\theta_n(t)$ of each waveguide. We note that the phase does depend on $t$ as well, but we will show that it is constant, to leading order. Plugging in our ansatz \cref{eq:cnansatz} into \cref{eq:standingwave}, we obtain the equation
\begin{align*}
e^{i \theta_n}&\left[ (\ddot a_n - a_n (\dot \theta_n)^2) 
+ i ( a_n \ddot\theta_n + 2 \dot a_n \dot \theta_n ) \right] \\
&+ k\left(e^{i\phi}a_{n-1}e^{i \theta_{n-1}} +e^{-i\phi}a_{n+1}e^{i \theta_{n+1}}\right)+|a_n|^2 a_n e^{i \theta_n} - \omega a_n e^{i \theta_n} = 0.
\end{align*}
where we have suppressed the dependence on $t$ and used the overdot notation for derivatives with respect to $t$ for convenience. Dividing by $e^{i \theta_n}$, this becomes
\begin{equation}\label{eq:st2}
\begin{aligned}
(\ddot a_n &- a_n (\dot \theta_n)^2) 
+ i ( a_n \ddot\theta_n + 2 \dot a_n \dot \theta_n )\\
&+ k\left(a_{n-1}e^{-i[(\theta_n - \theta_{n-1}) - \phi]} + a_{n+1}e^{i[(\theta_{n+1} - \theta_{n}) - \phi]} \right)+a_n^3 - \omega a_n = 0,
\end{aligned}
\end{equation}	
which we can split up into real and imaginary parts to get
\begin{align}
&\ddot a_n - a_n (\dot \theta_n)^2 +
 k\left(a_{n-1}\cos[(\theta_n - \theta_{n-1}) - \phi] + a_{n+1}\cos[(\theta_{n+1} - \theta_{n}) - \phi] \right)+a_n^3 - \omega a_n = 0 \label{eq:st2real} \\
&a_n \ddot\theta_n + 2 \dot a_n \dot \theta_n
+ k\left(-a_{n-1}\sin[(\theta_n - \theta_{n-1}) - \phi] + a_{n+1}\sin [(\theta_{n+1} - \theta_{n}) - \phi] \right) = 0. \label{eq:st2imag}
\end{align}
For $m$ even, numerical parameter continuation experiments suggest that there exist solutions with the following symmetries:
\begin{equation}\label{eq:symm}
\begin{aligned}
a_{-n}(t) &= a_{n}(t) && \qquad n = 1, \dots, m/2 \\
\theta_{-n}(t) &= -\theta_{n}(t) && \qquad n = 1, \dots, m/2 \\
\theta_0(t) &= 0 \\
\theta_{m/2}(t) &= 0.
\end{aligned}
\end{equation}
We verify in \cref{app:symm} that these symmetry conditions are consistent. WRITE all of the $a_n$ and $\theta_n$ as asymptotic series in powers of $k$. 
\begin{equation}\label{eq:basicseries}
\begin{aligned}
a_0(t) &= \psi(t) + \mathcal{O}(k) \\
a_n(t) &= \mathcal{O}(k) && n = 1, \dots, m/2 \\
\theta_n(t) &= n \phi + \mathcal{O}(k) && n = 1, \dots, m/2-1
\end{aligned}
\end{equation}

Since we are looking for a solution which is close to the NLS soliton in node 1 and close to 0 in the other nodes, our initial perturbation ansatz for the amplitudes is of the the form
\begin{align*}
a_1(t) &= \psi(t) + \mathcal{O}(k) \\
a_n(t) &= \mathcal{O}(k) && n = 2, \dots, M,
\end{align*}
where $\mathcal{O}(k)$ represents a power series in $k$ that I didn't write out. We do something similar for the angles. Plugging all of this in and equating like powers of $k$, we obtain the following asymptotic expansions for the amplitudes:
\begin{align*}
a_1(t) &= \psi(t) + k^2 \tilde{a}_1(t) + \mathcal{O}(k^3) \\
a_2(t) &= k \tilde{a}_2(t) + \mathcal{O}(k^2) \\
a_3(t) &= k^2 \tilde{a}_3(t) + \mathcal{O}(k^3) \\
&\vdots \\
a_M(t) &= k^{M-1} \tilde{a}_M(t) + \mathcal{O}(k^M),
\end{align*}
which we can write as
\begin{align*}
a_1(t) &= \psi(t) + k^2 \tilde{a}_1(t) + \mathcal{O}(k^3) \\
a_n(t) &= k^{n-1} \tilde{a}_n(t) + \mathcal{O}(k^n) && n = 2, \dots, M.
\end{align*}
Note that we have determined the magnitude (in $k$) of the leading order term for each amplitude $a_n(t)$, and that $a_n(t) = \mathcal{O}(k^{n-1})$, so the order of the amplitude increases with $n$. Similarly, for the angles, we obtain the asymptotic expansions
\begin{align*}
\theta_M(t) &= 0 \\
\theta_{M-1}(t) &= (M-1)\phi + k^2 \tilde{\theta}_{M-1} + \mathcal{O}(k^3) \\
\theta_{M-2}(t) &= (M-2)\phi + k^4 \tilde{\theta}_{M-2} + \mathcal{O}(k^5) \\
&\vdots \\
\theta_3(t) &= 2 \phi + k^{m-4} \tilde{\theta}_3(t) + \mathcal{O}(k^{m-3}) \\
\theta_2(t) &= \phi + k^{m-2} \tilde{\theta}_3(t) + \mathcal{O}(k^{m-1}) \\
\theta_1(t) &= 0.
\end{align*}
Note that in this case, the leading order term is always $\mathcal{O}(1)$, and the order of the next lowest-order term decreases with $n$. We can write this more simply as
\begin{align*}
\theta_n(t) &= (n-1)\phi + k^{m-2n+2}\tilde{\theta}_n(t) + \mathcal{O}(k^{m-2n+3}).
\end{align*}

The amazing thing is that we can actually obtain (recursive) formulas for the first remainder terms $\tilde{a}_n(t)$ and $\tilde{\theta}_n(t)$ in the asymptotic expansions above! These formulas can all be verified numerically. 

First, we solve for the lowest order terms in the expansion for the amplitudes. To do this, we use the real part of \cref{eq:st2} and ``work down'' from $\tilde{a}_2(t)$ to obtain the recursive formulas:
\begin{align*}
\tilde{a}_2(t) &= (\omega - \partial_t^2)^{-1} \psi(t) \\
\tilde{a}_3(t) &= (\omega - \partial_t^2)^{-1} \tilde{a}_2(t) \\
\tilde{a}_4(t) &= (\omega - \partial_t^2)^{-1} \tilde{a}_3(t) \\
&\vdots \\
\tilde{a}_{M-1}(t) &= (\omega - \partial_t^2)^{-1} \tilde{a}_{M-2}(t) \\
\tilde{a}_{M}(t) &= 2 \cos( \theta_{M-1}(t) + \phi )(\omega - \partial_t^2)^{-1} \tilde{a}_{M-1}(t),
\end{align*}
where we note that the operator $(\omega - \partial_t^2)$ is indeed invertible when $\omega > 0$ and we use either periodic, Dirichlet, or Neumann boundary conditions. We can write everything in terms of $\psi(t)$ to get the expressions
\begin{align*}
\tilde{a}_n(t)   &= (\omega - \partial_t^2)^{-(n-1)} \psi(t) && n = 2, \dots, M-1 \\
\tilde{a}_{M}(t) &= 2 \cos( \theta_{M-1}(t) + \phi )(\omega - \partial_t^2)^{-(M-1)} \psi(t).
\end{align*}
In addition, for node 1, the next lowest order term in the asymptotic expansion for $a_1(t)$ satisfies the equation
\[
(\partial_t^2 - \omega)\tilde{a}_1(t) + 3 \psi(t)^2 \tilde{a}_1(t) + 2 \tilde{a}_2(t) = 0.
\]

If we use only the leading order term from the asymptotic expansion for $\theta_{M-1}$, we have $\theta_{M-1} \approx (M-2) \phi = [(m/2)-1] \phi$, thus the equation for $\tilde{a}_{M}(t)$ becomes, to leading order,
\[
\tilde{a}_{M}(t) = -2 \cos( m \phi/2 )(\partial_t^2 - \omega)^{-1} \tilde{a}_{M-1}(t),
\]
which is 0 if we take $\phi = \pi/m$. This is the expected result. Note that this choice eliminates only the leading order term in the asymptotic expansion for $\tilde{a}_{M}(t)$; suppression may not complete, since there may still be terms which are higher order in $k$.

We can do better by solving for the remainder terms $\tilde{\theta}_n(t)$ in the asymptotic expansion of the angles. To do this, we use the imaginary part of \cref{eq:st2} and ``work up'' from $\tilde{a}_{M-1}(t)$ to get the recursive formulas:
\begin{align*}
\tilde{\theta}_{M-1}(t) &= -\sin(m\phi/2)\frac{ (\omega - \partial_t^2)^{-1} \tilde{a}_{M}(t)}{\tilde{a}_{M-1}(t)} \\
\tilde{\theta}_{M-2}(t) &= \frac{ (\omega - \partial_t^2)^{-1} \left( \tilde{a}_{M-1}(t) \tilde{\theta}_{M-1}(t) \right) }
{\tilde{a}_{M-1}(t)} \\
&\vdots \\
\tilde{\theta}_{3}(t) &= \frac{ (\omega - \partial_t^2)^{-1} \left( \tilde{a}_{4}(t) \tilde{\theta}_{4}(t) \right) }
{\tilde{a}_{3}(t)} \\
\tilde{\theta}_{2}(t) &= \frac{ (\omega - \partial_t^2)^{-1} \left( \tilde{a}_{3}(t) \tilde{\theta}_{3}(t) \right) }
{\tilde{a}_{2}(t)},
\end{align*}
which we can write as
\begin{align*}
\tilde{\theta}_{M-1}(t) &= -\sin(m\phi/2)\frac{ (\omega - \partial_t^2)^{-1} \tilde{a}_{M}(t)}{\tilde{a}_{M-1}(t)} \\
\tilde{\theta}_{n}(t) &= \frac{ (\omega - \partial_t^2)^{-1} \left( \tilde{a}_{n+1}(t) \tilde{\theta}_{n+1}(t) \right) }
{\tilde{a}_{n}(t)} && n = 2, \dots, M-2.
\end{align*}
Now that we have these expressions, we substitute the leading order expression for $\tilde{\theta}_{M-1}(t)$ into the expression for $\tilde{a}_{M}(t)$ to get
\begin{align*}
\tilde{a}_{M}(t) &= 2 \cos[(M-1)\phi + \mathcal{O}(k^2) ] (\omega - \partial_t^2)^{-1} \tilde{a}_{M-1}(t) \\
&= 2 \cos(m \phi/2) (\omega - \partial_t^2)^{-1} \tilde{a}_{M-1}(t) + \mathcal{O}(k^2).
\end{align*}
Substituting this into the equation for $\tilde{\theta}_{M-1}(t)$, we get
\begin{align*}
\tilde{\theta}_{M-1}(t) &= -2 \sin(m \phi/2)\cos(m \phi/2)\frac{ (\partial_t^2 - \omega)^{-2} \tilde{a}_{M-1}(t)}{\tilde{a}_{M-1}(t)} + \mathcal{O}(k^2) \\
&= -\sin(m \phi)\frac{ (\partial_t^2 - \omega)^{-2} \tilde{a}_{M-1}(t)}{\tilde{a}_{M-1}(t)} + \mathcal{O}(k^2).
\end{align*}
When $\phi = \pi/m$, we have $\tilde{\theta}_{M-1}(t) = \mathcal{O}(k^2)$, which eliminates the lowest order remainder term in the expression for $\theta_{M-1}(t)$. Using the recursive formulas above, this eliminates the lowest remainder term for all of the angles $\theta_n(t)$.

\appendix

\section{Symmetries}\label{app:symm}

We verify that the symmetry relations in \cref{eq:symm} are consistent. First, for $n = 2, \dots, m/2-1$, we take equation \cref{eq:st2} for $-n$, substitute the symmetries for $a_n$ and $\theta_n$ from \cref{eq:symm}, and simplify, to obtain 
\begin{equation*}
\begin{aligned}
(\ddot a_n &- a_n (\dot \theta_n)^2) 
- i ( a_n \ddot\theta_n + 2 \dot a_n \dot \theta_n )\\
&+ k\left(a_{n-1}e^{i[(\theta_n - \theta_{n-1}) - \phi]} + a_{n+1}e^{-i[(\theta_{n+1} - \theta_{n}) - \phi]} \right)+a_n^3 - \omega a_n = 0,
\end{aligned}
\end{equation*}	
which is the complex conjugate of \cref{eq:st2} for $n$. For $n = 0$, we take equation \cref{eq:st2} for $0$, substitute the symmetries for $a_n$ and $\theta_n$ from \cref{eq:symm}, and simplify, to obtain 
\begin{equation*}
\begin{aligned}
(\ddot a_0 &- a_0 (\dot \theta_0)^2) 
+ i ( a_0 \ddot\theta_0 + 2 \dot a_0 \dot \theta_0 )
+ 2 k a_1 \cos(\theta_1 - \phi)(\cos \theta_0 + i \sin \theta_0) + a_n^3 - \omega a_n = 0.
\end{aligned}
\end{equation*}
The imaginary part is
\begin{equation}\label{eq:n0imagpart}
a_0 \ddot\theta_0 + 2 \dot a_0 \dot \theta_0 = 2 k a_1 \cos(\theta_1 - \phi) \sin \theta_0,
\end{equation}
for which $\theta_0(t) = 0$ is a solution, thus this condition is consistent. Following the same procedure by using the imaginary part of equation \cref{eq:st2}, we can show that the condition $\theta_{m/2}(t) = 0$ is consistent.

\section{Asymptotics}

Let $a_m(t)$ and $\theta_m(t)$ be the amplitudes and phases at site $n$, as in the ansatz \cref{eq:cnansatz}, where $n \in S = \{ 0, \pm 1, \dots, \pm m/2-1, m/2 \}$. We will take 
\[
a_m(t), \theta_m(t) \in H^2(\R) \subset L^2(\R),
\]
where $L^2(\R)$ is the space of real, valued square-integrable functions on $\R$ equipped with the standard inner product, and $H^2(\R)$ is the corresponding Sobelev space. We note that the system of equations \cref{eq:st2} is translation invariant, i.e. if $\{ a_n(t), \theta_n(t)\}_{n\in S}$ is a solution, then so is $\{ a_n(t-\tau), \theta_n(t-\tau)\}_{n\in S}$ for any $\tau \in \R$. When $k = 0$, we want the solution $a_0(t)$ to be the ordinary NLS soliton $\psi(t)$, which is centered at 0. To ensure that the solution we find is not shifted in $t$, we impose the phase condition on the amplitude $a_0(t)$
\begin{equation}\label{eq:phasecond}
\langle a_0(t), \dot{\psi}(t) \rangle_{L^2(\R)} = \int_{-\infty}^\infty a_0(t) \dot{\psi}(t) dt = 0.
\end{equation}
In other words, $a_0(t)$ will have no component in the direction of $\dot{\psi}(t)$.

Next, define the linear operator $\Lw: H^2(\R) \subset L^2(\R) \rightarrow L^2(\R)$ by
\begin{equation}\label{eq:Lw}
\Lw = \omega - \partial_t^2.
\end{equation}
Since all functions in $L^2(\R)$ vanish at infinity, the kernel of $\Lw$ is $\{0\}$, i.e. the only solution in $H^2(\R)$ to $\Lw f = 0$ is $f = 0$. Furthermore the operator $\Lw$ is invertible [REF?].

Finally, we recall that the NLS soliton $\psi(t)$ is a real-valued, standing wave solution to the NLS equation with frequency $\omega$, thus it solves 
\begin{equation}\label{eq:NLSreal}
\ddot{u} + u^3 - \omega u = 0. 
\end{equation}
Linearizing this equation about $\dot{\psi}(t)$ yields the self-adjoint linear operator $\calL(\psi): H^2(\R) \subset L^2(\R) \rightarrow L^2(\R)$, defined by
\begin{equation}\label{eq:Lpsi}
\calL(\psi) = \partial_t^2 - \omega + 3 \psi^2.
\end{equation}
The kernel of $\dot{\psi}(t)$ is one-dimensional, and is spanned by $\{ \dot{\psi}(t) \}$. By the Fredholm alternative [REF], since $\calL(\psi)$ is self-adjoint, the equation $\calL(\psi) = u(t)$ has a solution if $u(t) \perp \dot{\psi}(t)$.

\subsection{Node 0}

For node 0, the site in which the maximum intensity is concentrated, we take the power series ansatz 
\[
a_0(t) = \psi(t) + k a_0^{(1)}(t) + k^2 a_0^{(2)}(t) + \mathcal{O}(k^3)
\]
for the amplitude, where $\psi(t)$ is the NLS soliton \cref{eq:NLSsoliton}. For the phase, we take $\theta_0(t) = 0$, which we showed in \cref{app:symm} satisfies the imaginary part \cref{eq:n0imagpart} of equation \cref{eq:st2} for $n=0$. In addition, from \cref{eq:basicseries}, we have
\[
a_1(t) = \mathcal{O}(k), \qquad \theta_1(t) = \phi + \mathcal{O}(k).
\]
Substituting this into the $n=0$ equation of \cref{eq:st2real}, using the Taylor series expansion $\cos(\theta_1-\phi) = \mathcal{O}(k^2)$, collecting powers of $k$, and simplifying, we get
\begin{equation*}
\begin{aligned}
&\left(\ddot{\psi} + \psi^3 - \omega \psi\right) 
+ k\left(\ddot a_0^{(1)} - \omega a_0^{(1)} + 3 \psi^2 a_0^{(1)}\right) \\
&\qquad\qquad+ k^2\left(\ddot a_0^{(2)} - \omega a_0^{(2)} + 3\left(a_0^{(1)}\right)^2 + 3 \psi^2 a_0^{(2)}\right) + \mathcal{O}(k^3) = 0,
\end{aligned}
\end{equation*}
where we have dropped the dependence on $t$ for simplicity. The $\mathcal{O}(1)$ term is 0 since $\psi$ solves \cref{eq:NLSreal}. The $\mathcal{O}(k)$ term can be written as $\calL(\psi)a_0^{(1)}=0$, where $\calL(\psi)$ is defined by \cref{eq:Lpsi}. Since the kernel of $\calL(\psi)$ is spanned by $\dot \psi(t)$, $a_0^{(1)} = c \dot \psi(t)$ for some constant $c$. However, since we wish our solution to satisfy the phase condition \cref{eq:phasecond}, we will take $c = 0$, from which it follows that $a_0^{(1)} = 0$. The $\mathcal{O}(k^2)$ term then becomes
\[
\ddot a_0^{(2)} - \omega a_0^{(2)} + 3 \psi^2 a_0^{(2)} + 2 a_1^{(1)} = 0, 
\]
which we can write as 
\[
\calL(\psi) a_0^{(2)} = -2 a_1^{(1)}.
\]
Once we determine $a_1^{(1)}$, which we will do in the next step, we can solve for $a_0^{(2)}$, provided $a_1^{(1)} \perp \dot\psi$. Putting all of this together, we have the expression for $a_0$
\begin{equation}\label{eq:a0eq}
a_0(t) = \psi(t) + k^2 a_0^{(2)}(t) + \mathcal{O}(k^3) = \psi(t) + \mathcal{O}(k^2)
\end{equation}

\subsection{Amplitudes}
We can now determine the leading order terms for the amplitudes $a_n$, for $n = 1, \dots, m/2$, using the real parts \cref{eq:st2real} of the equations \cref{eq:st2}. The phases $\theta_n$ will be determined in the next step. For $n=1$, we take the power series ansatz 
\[
a_1(t) = k a_1^{(1)}(t) + \mathcal{O}(k^2).
\]
From \cref{eq:basicseries}, we have
\[
a_2(t) = \mathcal{O}(k), \qquad \theta_1(t) = \phi + \mathcal{O}(k), \qquad \theta_2(t) = 2 \phi + \mathcal{O}(k),
\]
and we note that $\dot \theta_1(t) = \mathcal{O}(k)$. Substituting these together with the expression \cref{eq:a0eq} for $a_0(t)$ into the $n=1$ equation of \cref{eq:st2real}, expanding the cosine terms in Taylor series, collecting powers of $k$, and simplifying, we obtain the equation
\[
k\left(\ddot a_1^{(1)} - \omega a_1^{(1)} + \psi\right) + \mathcal{O}(k^2) = 0.
\]
% We note that the nonlinear term $a_1^3$ does not contribute to the lowest order term in the asymptotic expansion. 
We use the $\mathcal{O}(k)$ term to solve for $a_1^{(1)}$ to get 
\begin{equation}\label{eq:a11}
a_1^{(1)}(t) = (\omega - \partial_t^2) \psi(t),
\end{equation}
which has a solution since $\Lw$ is invertible. This gives us
\begin{equation}\label{eq:a1eq}
a_1(t) = k a_1^{(1)}(t) + \mathcal{O}(k^2) = k (\omega - \partial_t^2) \psi(t) + \mathcal{O}(k^2).
\end{equation}
We continue this process iteratively. For $n = 2$, we take the power series ansatz 
\[
a_2(t) = k a_2^{(1)}(t) + k^2 a_2^{(2)}(t) + \mathcal{O}(k^3),
\]
and use
\[
a_3(t) = \mathcal{O}(k), \qquad \theta_2(t) = 2 \phi + \mathcal{O}(k), \qquad \theta_3(t) = 3 \phi + \mathcal{O}(k)
\]
from \cref{eq:basicseries}. Substituting these together with the expression \cref{eq:a1eq} for $a_1(t)$ into the $n=2$ equation of \cref{eq:st2real} and following what we did above, we obtain the equation
\begin{align*}
k\left(\ddot a_2^{(1)} - \omega a_2^{(1)}\right) + 
k\left(\ddot a_2^{(2)} - \omega a_2^{(2)} + a_1^{(1)} + a_3^{(1)}\right) +\mathcal{O}(k^3) = 0.
\end{align*}
From the $\mathcal{O}(k)$ term, $\Lw a_2^{(1)} = 0$, thus since the kernel of $\Lw$ is $\{ 0\}$, we must have $a_2^{(1)} = 0$. We then use the $\mathcal{O}(k^2)$ term to solve for $a_2^{(2)}$ to get 
\begin{equation}\label{eq:a22}
a_2^{(2)}(t) = (\omega - \partial_t^2)\left(a_1^{(1)} + a_3^{(1)}\right),
\end{equation}
where we found $a_1^{(1)}$ in the previous step. In the next step, we will determine that $a_3^{(1)} = 0$. Since $\Lw$ is invertible. This gives us
\begin{equation}\label{eq:a2eq}
a_1(t) = k^2 a_2^{(2)}(t) + \mathcal{O}(k^3) = k^2 (\omega - \partial_t^2) \left(a_1^{(1)} + a_3^{(1)}\right) + \mathcal{O}(k^3).
\end{equation}






\end{document}