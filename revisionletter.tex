\documentclass[11pt]{letter}

\usepackage[hmargin={1.0in,1.0in},%
            vmargin={1.0in,1.0in},%
            nohead,%
            nofoot,%
            ]{geometry}                                 % the page layout without fancyhdr
\pagestyle{empty}

\begin{document}
\address{Ross Parker \\
Department of Mathematics \\
Southern Methodist University \\
Dallas, TX 75275 \\
\texttt{rhparker@smu.edu}}%
\signature{Ross Parker}
\begin{letter}{Editor, Studies in Applied Mathematics}

\opening{Dear Editor,}

On behalf of my co-authors, Yannan Shen, Alejandro Aceves, and John Zweck, I would like to submit revisions to the article ``Spatiotemporal dynamics in a twisted, circular waveguide array'' for consideration of publication in Studies in Applied Mathematics. We are grateful to the referees for their careful reading of the original manuscript, and their comments and suggestions regarding how we could improve it. All the suggestions for improvement have been systematically taken into careful consideration and incorporated into the revision, as noted below. The portions of the manuscript which have been revised or added are indicated using red text. Given the improvements made in accordance with the requests of the referees, we hope that you will now find the manuscript to be suitable for publication. We will be sincerely looking forward to your editorial decision.

Associate editor 2:
\begin{enumerate}
    \item \emph{The list of references does not include the pioneering papers where the method picked by the authors was developed: Physica D, 212, 1-19 (2005); Physica D, 212, 20-53 (2005)} These two references are now cited in the introduction.
    \vspace{4mm}

    \item \emph{The recent developments are not cited: Phys. D 370 (2018), 1–13;  Phys. D 408 (2020), 132414, 21 pp} We mention that the model without temporal dispersion reduces to DNLS on a finite lattice when twist parameter $\phi = 0$, and cite these recent developments.
    \vspace{4mm}
\end{enumerate}

Reviewer 1:
\begin{enumerate}
    \item \emph{Did the authors try to calculate terms other than the leading one of $\tilde{a}_{N/2}$, in order to support the numerical finding that the intensity in this node is completely suppressed?} We now include a discussion of higher order terms in the asymptotic expansion, from which we obtain the supression result to a few more orders of magnitude. Although it may be possible to continue the asymptotic analysis to higher orders, the terms involved become increasingly more unwieldy, as we discuss briefly in Appendix B.4.
    \vspace{4mm}

    \item \emph{In equation (15) how is $\tilde{a}_0$ calculated? perhaps I am missing something.} We have moved the equation for $\tilde{a}_0$ up for clarity. It is now equation (16). 
    \vspace{4mm}

    \item \emph{In figure 4 do the lines with $\phi\neq0.5$ possess a maximum or do the tend to infinity?}$k$ reaches a maximum at this point. This is stated more clearly in the text.
    \vspace{4mm}

    \item \emph{In Figure 5 (e) the authors could mention that the line for node 3 lies at the margin of the numerical calculations error.} We now mention this in the text in reference to this figure.
    \vspace{4mm}
\end{enumerate}

Reviewer 2:
\begin{enumerate}
    \item \emph{Regarding the particular choice of the number of $N = 6$ oscillators, I think that some further justification should be given for this choice. While it is written that a consideration of large number of N makes a potentially, non-tractable increase of the computational efforts required regarding the numerical codes, I think a second example for a number N of oscillators which can be handled by the codes would be useful.} We now show the suppression effect occurs for $N=12,18,24$ (Figure 6c).
\end{enumerate}

\closing{Sincerely,}

\end{letter}
\end{document}